\documentclass[12pt]{article}
\usepackage[top=1in, bottom=1in, left=1.3in, right=1.3in]{geometry}

\usepackage{hyperref}
\begin{document}
\author{Dionisio Perez-Mavrogenis}
\date{\today}
\title{FixMe User Testing}
\clearpage
\maketitle
\thispagestyle{empty}
\pagebreak

\section*{Abstract}
This is the user testing specification for the FixMe framework. It is partitioned into three parts, one for each component, with appropriate introductions and scenarios outlined in each section \\

Please proceed by reading the Introduction and General Guidelines sections, and then to the table of contents to find your appropriate section.

\pagebreak
\tableofcontents
\pagebreak

\section{Introduction}
The purpose of your being subjected to these tests is purely to draw conclusions and iterate over the design of the various components and user interfaces of the framework in an attempt to improve the user experience.\\

The data collected by this study will be confidential and will be used for no other purpose than the improvement of the framework. \\

The only personal information that will be required by regarding you as an individual is your age, name, email address and current occupation. None of these details will be forwarded outside of the university of Southampton and you will be notified for any disclosure of your details to anybody. Your details will serve as a medium of providing authenticity of the results, as well as a means to contact you for follow-up questions.\\

\textbf{You are not allowed to share solutions with other candidates taking part on the study, should you happen to know one.}\\

\textbf{Please answer all questions at least to the extent of detail of the question being asked. It is alright if you use slang, abbreviations or whatever suits you as long as your message is propersly conveyed. }\\

\textbf{By continuing with the study you confirm that you have read, understood and agreed to the conditions and obligations present on this document.} \\

\textbf{Please direct any questions you have to \texttt{dpm3g10@ecs.soton.ac.uk}, with the subject line \texttt{FixMe USER TESTING}}.
\pagebreak

\section{General Guidelines}
On each test section you will be assuming a role (user, moderator, administrator) and you will be given brief introductions as to what you are supposed to do, as well as some exercises for you to do, little tasks that should be easily accomplished and should take little time. \\

After you have completed your tasks, you will need to submit a paragraph of your overal experience. Please, try to at least touch on the following tasks as this is what we would like to get feedback on:

\begin{itemize}
\item[Task difficulty]\hfill \\How hard was it for you to accomplish each task? How much time did it take you to complete it? Was anything frustrating or getting in your way?\\
\item[Interface looks]\hfill \\ Do you like the current layout of the application ? Would you like something to change (remove, add, redesign or anything similar) ?\\
\item[Iterface familiarity] \hfill \\ How familiar was the interface to you? Did you have trouble navigating or finding your way around ? Did it remind you of something you have already used ? What would you change ?\\
\item[Your feelings] \hfill \\
Finally, I would REALLY like to hear your opinion. How did you feel ? Was the experience frustrating in any way ? What would you change about the process ?
\item[Your additions] \hfill \\
This is where you express your ideas and opinions or feelings on something that is not covered above, or your general views and opinions! 
\end{itemize}


Although this might vary, these tests shouldn't take long. Finding out what to do and overally completing the test should not take you more than 30 minutes, and that is if you work really slow (nobody likes to be rushed!) !\\

If you don't understand how to do something, or the wording is unclear to you, please don't hesitate to contact me. It's not your fault. It is most likely me not wording terms and concepts properly, or the layout of the application that is in primal stages yet!

\pagebreak
\section{FixMe User Application}
\subsection{Your Personal}

You are a member of some sportsclub which has got lots of facilities and equipment. While you do enjoy the services and variety of activities, you find that every now and again something breaks and you have to wait some time before it gets fixed. \\

You think the time it takes for the maintenance team to respond is more than you would like to wait (or find reasonable), and would like that to change. You are aware that the administration of the facilities that you are using want to change that as well and are trying to get people to use a new framework called "FixMe" to help them with their maintenance.\\

The way the framework works is that users can now install an application on their smartphones and give them the ability to report a problem as they spot it. You would like to take part in that and install the application on your phone so you can help with with the maintenance process and therefore suffer less disruption yourself from broken equipment.\\

To help get you started, the administrators have printed out QR Codes for installing the application on your phone, as well as print a couple of pages to give you detailed information on the application, should you need it!

This is what they told you

\begin{quotation}
"\dots installing the application on your phone is really easy! You just have to scan this QR code here \href{https://build.phonegap.com/apps/347350/share}{https://build.phonegap.com/apps/347350/share} and it will download and install the application on your phone! There is more information on the first section of the guide we gave you\footnote{You can download it here : \href{http://dl.dropbox.com/u/23023752/user.pdf}{http://dl.dropbox.com/u/23023752/user.pdf}} ! \dots "
\end{quotation}

Suppose that the FixMe Code that the organization you do an activity with gave you was "abc".

\pagebreak
\subsection{Your Tasks}
These are the tasks you are required to complete! \\
\textbf{Hint:} Please keep a text editor handy, as you will need to write some short answers down soon!
\begin{enumerate}

\item Preliminaries\\
About QR Codes...
\begin{enumerate}
\item Do you know what a QR code is? Have you heard the term before?
\item Do you know how to use a QR code?
\item Do you think you can easily find a QR code reader application for your phone, should you not have one already?
\item How annoyed are you that you might have to install a QR reader to install the FixMe user application?
\end{enumerate}

\item Intalling the application\\
Here you will be installing the application! Please proceed to read the first section of the guide only!
\begin{enumerate}
\item Are the instructions in the guide clear enough for you to understand?
\item Would you like some more information on this part?
\item Would you change anything to make that part of the guide more helpful?
\end{enumerate}

\item The login prompt\\
Assuming you have successfully managed to install the application, you fire it up and are presented with the log in screen!
\begin{enumerate}
\item What do you see? What do you think each component is there for?
\item What information is being asked in the FixMe Code field? where can you find that information?
\item Can you login? If so, go ahead! If no, what went wrong?
\end{enumerate}

\item You are in the main menu, finally!\\
You are looking at this new software!
\begin{enumerate}
\item What do you think each button does?
\item Is something confusing you?
\item Have a look around, open the different menus and explore tha application!
\end{enumerate}

\item As you work out, a wild breakage appears!\\
You decide to use your FixMe application!
\begin{enumerate}
\item Do you know where cou have to go to submit a new report? If so, go there!
\item Having opened the new report form, what do you think each component is about?
\item Is it possible to attach a photograph with a report?
\item Is it possible to indicate the location of the report on the map?
\item Do you understand what submitting anonymously or not is?
\item What kind of information do you need to enter, and would you use it?
\end{enumerate}

\item Viewing reports!\\
Have a look at some of the screens from the main menu!
\begin{enumerate}
\item What do you think "Favourites, My History and Most Recent" are all about? What would you expect to see filed under each?
\item When you are in these screens, what do you think each button does?
\item What would you have to do to view a list of the most recent reports?
\item How can you mark a report as a favourite, and why would you do that?
\item How would you view the full details of a report?
\end{enumerate}

\item Report Details\\
When viewing the details of a report
\begin{enumerate}
\item What information do you see? Is anything unclear to you?
\item What other things can you find to do?
\end{enumerate}

\item Application settings!\\
Go into the settings menu please!
\begin{enumerate}
\item What do you see here? How do you think this information is used?
\item Can you find any other things to do?
\item Do you think you can wipe out reports?
\end{enumerate}

\item Multiple facilities\\
You are so active, that your evenings are filled with activities! As a result, you visit multiple establishments, but you find that both have mulfunctioning equipment! Thatnkfully, both sites are using FixMe!
\begin{enumerate}
\item What would you have to do to report breakages to the second sports club?
\end{enumerate}

\item Help on the phone!\\
Do you think that a help section on the application would be valuable? If so,
\begin{enumerate}
\item What was unclear to you about the application?
\item Did you have read the FAQ given to you to understand something?
\item What would you like to see on the phone that would make the user's life easier?
\end{enumerate}

\item Final thoughts\\
What are your overall impressions regarding the application?
\begin{enumerate}
\item Was it familiar to you? Did it remind you on anything you've used before?
\item How familiar was the layout of the application?
\item Would you change anything (either in the layout or the content) about the application?
\item Was there any other information you would like to have on the phone?
\item Was there any additional functionality you found that was missing, but would like to see on the application?
\end{enumerate}

\end{enumerate}

\vspace{5em}
\textbf{Congratulations, you are done!} If you are reading this, you are very brave indeed!\\

Please send the text file with your answers to \texttt{dpm3g10@ecs.soton.ac.uk}, not forgetting to include how much time you needed to complete your quest, how difficult you found the overall experience and how fun, boring, exciting or painful you found the experience overall, along with your other special comments if you have any!

\pagebreak
\section{FixMe Moderator Application}
\subsection{Your Persona}
You will assume the role of a person working in a maintenance team for an organisation. You have a smartphone provided by your employer, to assist you in work-related communication. It has lately been noticed that it takes some time for your team to fix broken stuff around, but it is not really your fault, as you can not spot everything in time yourselves by just going around and looking for stuff! \\

So congratulations! If you are reading this, it means that your boss thinks you are suitable to handle a new piece of technology deployed on your organisation!\\

This new technology is nothing more than another application on your smartphone. The application is small and simple, and can help you view what you have to do wherever you are, so you don't need to be randomly wandering around to find things to do!\\

Your boss has given you the following general introduction on the new framework \\
\begin{quotation}
\emph{"\dots the new framework we will be using works like this : the users have an application to their phones to submit to us reports about stuff not working well and these reports are sent to a web server and stored there. With the application you have on your phones, you will be able to see new reports and issues as they are reported, in an attempt to help you do your job a bit easier. We believe that the following technology will improve how we operate and that's why we are deploying it\dots "}
\end{quotation}

He has printed out a couple of pages for you to read and get you started on using the moderator application\footnote{You can download that guide here : \href{http://dl.dropbox.com/u/23023752/mod.pdf}{http://dl.dropbox.com/u/23023752/mod.pdf}}, but you don't want to read that! You were elected to lead, not to read!\\

You might need to read it later though, so don't throw it away!\\

Off we go!\\

Note : Assume that your boss gave you a username:Janitor and password:Janitor1 to use with the application.

\pagebreak
\subsection{Your tasks}
These are the tasks you are required to complete! \\
\textbf{Hint:} Please keep a text editor handy, as you will need to write some short answers down soon!

\begin{enumerate}
\item Starting off; Install the moderator application on your phone\footnote{Url here : \href{https://build.phonegap.com/apps/347344/share}{https://build.phonegap.com/apps/347344/share}}! You might want to read the first section of the FAQ your boss gave you to make it easy!
\begin{enumerate}
\item Did you manage to install it? If yes, was it hard? If not, what went wrong?
\item Fire up the application!
\end{enumerate}

\item The log in screen!
\begin{enumerate}
\item Do you understand what is being asked from you to provide?
\item Do you have that information?
\item Can you login? If not, what went wrong?
\end{enumerate}

\item You are in the main menu! Without reading the manual,
\begin{enumerate}
\item What do you see? 
\item What do you think each button does?
\item Do the icons appropriately describe what you will see when you press them?
\item Do you see anything confusing?
\end{enumerate}

\item Viewing reports\\
It is a new day at work, and you want to get started using your smartphone!
\begin{enumerate}
\item How can you update your phone with the newest reports?
\item How can you see what your boss has assigned to you do?
\item Is there a way in which you can see where you are on the map and where the problems are?
\end{enumerate}

\item Maintaining your phone\\
You are noticing your phone slowing down a bit, and you might think that it is because of all those reports you have solved!
\begin{enumerate}
\item Do you think you can wipe out some of the reports? If so, how?
\item Have a look at the setting menu.
\item What do you see now, and what do you think these buttons do?
\item Would you press those delete buttons or do they look scary?
\end{enumerate}

\item Looking at reports\\
You have downloaded the reports your boss have assigned to you, and would like to know what the breakage is on a given report.
\begin{enumerate}
\item Go to the assigned reports view and download the latest assigned reports.
\item Viewing the list of reports, do you think you can search for specific terms in reports without opening all reports?
\item What would you have to do to see the full details of a report?
\end{enumerate}

\item Report Details\\
You have successfully opened up a report and are viewing its full details!
\begin{enumerate}
\item What information do you see?
\item Is that information enough, or would you like to see something more?
\item Can you spot where that issue occurs on the map?
\item Would you change something? If so, what and why?
\end{enumerate}

\item Reporting fixed issues \\
You have been assigned report X and have successfully fixed it!
\begin{enumerate}
\item What should you do now that you fixed it?
\item How would you let others(the system included) know that you fixed it?
\end{enumerate}

\item Working for other people\\
You are so good in what you do, that another company employed you as a maintenance engineer! Turns out they are using the FixMe framework as well!
\begin{enumerate}
\item You now work for two different employers. How many sets of credentials do you have (a set of credentials is a "Moderator Username-FixMe Code-Password" combination)?
\item How can you switch in order to view the new company's reports?
\end{enumerate}

\item Impressions?\\
All this new technology could be stressful!
\begin{enumerate}
\item How do you feel about the system?
\item How difficult or easy is it to use? Any frustration?
\item How complicated is it for you?
\item Are any of the terms confusing?
\item What would you change to make it more familiar?
\end{enumerate}

\item Reading the FAQ\\
Give that guide your boss gave you a quick read!
\begin{enumerate}
\item Was it hard to understand?
\item Was something not clear or difficult to remember?
\item Please list the most important things you can do, as per the guide!
\end{enumerate}

\item Help on the phone!\\
Do you think that a help section on the application would be valuable? If so,
\begin{enumerate}
\item What would you like to see there?
\item What tasks or things weren't explained well on the FAQ that your boss gave you, and might also be needed on the phone?
\end{enumerate}

\item Final thoughts\\
What are your overall impressions regarding the application?
\begin{enumerate}
\item Was it familiar to you? Did it remind you on anything you've used before?
\item How familiar was the layout of the application?
\item What would you change (either in the layout or the content) about the application?
\item Was there any other information you would like to have on the phone?
\end{enumerate}

\end{enumerate}


\vspace{5em}
\textbf{Congratulations, you are done!} If you are reading this, you are very brave indeed!\\

Please send the text file with your answers to \texttt{dpm3g10@ecs.soton.ac.uk}, not forgetting to include how much time you needed to complete your quest, how difficult you found the overall experience and how fun, boring, exciting or painful you found the experience overall, along with your other special comments if you have any!

\pagebreak
\section{FixMe Web Interface Application}
\subsection{Your persona}
Imagine you are the boss, or working for a boss that trusts you completely and allows you to deal with problems however you like. \\

You are in charge of certain facilities and establishments, filled with equipment which your clients use! Think of something like tennis or basketball courts, or a large car park or even a set of playgrounds for kids. \\

You want to keep all that equipment working and therefore you are in charge or work within a maintenance team that handles broken things (like torn basketball nets or broken slides or broken lamps). \\

However, you know that your staff won't like it if you tell them to constantly go around checking for broken stuff to use and you know that this would be inefficient!\\

You therefore need a mechanism for keeping track of the state of things and your facilities with the minimum amount of effort. An idea came to your head ! What if you could allow your users to report whatever issue they spot to you ?\\

You discuss your idea with your boss (if any) and your colleagues, and they tell you that a system called FixMe that might suit what you need is out there!\\

\textit{Don't be afraid to add or delete things, these can always be recreated or cleaned up later!}\\

Note : When you need to see a system with some reports, you can login with the credentials username:dio, password:dio, FixMe Code:abc.

\pagebreak
\subsection{Your tasks}
Your sample tasks! Start your internet quest here $\rightarrow$  \href{http://fixme.be}{\texttt{http://fixme.be}}\\
\textbf{Hint:} you might want to open a text editor now, for writing down some answers to the following exercises!

\begin{enumerate}

\item Find out what this "FixMe framework" is all about!
Before registering, 
\begin{enumerate}
\item do you generally understand what it is?
\item can you describe in general how it works?
\item what can you do on the website at this stage?
\end{enumerate}

\item Can you create your account? If so, do it!

\item Have a look around the website, see what you can find!\\
Please write down answers to the following :
\begin{enumerate}
\item Write down where you started from!
\item Do you know what the framework is now?
\item Does it fit your purpose?
\item Where would you look to get more information on something? How can you get there?
\end{enumerate}

\item Have some fun with the reports!\\
Please write down the answers to the following :
\begin{enumerate}
\item Find out a way to view all unresolved reports!
\item What can you do here with a reports?
\item When seeing a listing of all the reports, what information can you see ? Do you think it is enough?
\item Please describe briefly what you think each column represents!
\end{enumerate}

\item View the full details of any report ! \\
Please write down the answers to the following :
\begin{enumerate}
\item What information do you see?
\item What things can you do here? What do you think each one does?
\item What is the status of that report?
\end{enumerate}

\item Time for moderator fun!\\
Please write down the answers to the following :
\begin{enumerate}
\item what do moderators do in general, could you explain that to in a sentence or two?
\item Do you have any moderators? How can you tell?
\item How can you add a moderator to assist you?
\item If one of your maintenance people stops working for you, should he still be a moderator? Can you change that? If so, change it!
\item After you have added a moderator, how can he help? How would you give him work to do?
\end{enumerate}

\item Being in charge of your users!\\
Please write down the answer to the following :
\begin{enumerate}
\item As you arrive to the users page, explain what you see!
\item What is that information? Can you understand what that information represents?
\item What happens if you click on one of the items in the list?
\item Can you tell me what that information is?
\item Can you stop a user from posting wrong reports? If so, how?
\item Can you clear all of the reports of a user? If so, how?
\item What other things can you do in this page?
\item If you mistakenly banned a user, is it possible to unban him? If so, what would you need to do?
\end{enumerate}

\item Your account\\
Please write down the answers to the following :
\begin{enumerate}
\item Do you know how to manage your account?
\item What information can you change?
\end{enumerate}

\item Educating your users\\
Please write down the answers to the following :
\begin{enumerate}
\item Where can you find information to give to your users if they ask something about the user application?
\item Can you get your maintenance staff to read some document to educate them more on how their application works? If so, what would that be?
\item Could you do the same for the users?
\item If you where to find that information, can you think of a suitable way to hand it out?

\end{enumerate}

\end{enumerate}

\vspace{5em}
\textbf{Congratulations, you are done!} If you are reading this, you are very brave indeed!\\

Please send the text file with your answers to \texttt{dpm3g10@ecs.soton.ac.uk}, not forgetting to include how much time you needed to complete your quest, how difficult you found the overall experience and how fun, boring, exciting or painful you found the experience overall, along with your other special comments if you have any!

\end{document}